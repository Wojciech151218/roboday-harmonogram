\documentclass{article}
\usepackage[utf8]{inputenc}
\usepackage{polski}
\usepackage{geometry}
\geometry{a4paper, margin=1in}
\begin{document}
\begin{center}
\Large\textbf{Harmonogram RoboDay}
\large\textbf{Zespół Szkół Łączności}
\end{center}
\vspace{1cm}
\begin{center}
\begin{tabular}{|l|l|}
\hline
\textbf{Wydarzenie} & \textbf{Godzina} \\
\hline
Rozpoczęcie RoboDay & 10.00-10.15 \\
\hline
Wykład : PP i WIiT & 10.15.-10.30 \\
\hline
Informacje o konkursach & 10.30-10.40 \\
\hline
Wykład : CPU vs GPU vs TPU vs DPU vs QPU, czyli czym jest to PU (Processing Unit) & 10.40.-10.55 \\
\hline
Laboratoria : wyświetlacz 7 segemntowy & 11.00-11.20 \\
\hline
Oglądanie robotów & 11.20-11.40 \\
\hline
Wycieczka & 11.40-12.00 \\
\hline
Konkurs indywidualny & 12.00-12.45 \\
\hline
Konkurs indywidualny & 12.00-12.45 \\
\hline
Konkurs zespołowy & 12.00-12.45 \\
\hline
Wykład : Jak uczyć kreatywności? & 12.45-13.00 \\
\hline
Zakończenie i rozdanie nagród & 13.00-13.10 \\
\hline
\end{tabular}
\end{center}
\end{document}