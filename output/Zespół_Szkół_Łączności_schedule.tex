\documentclass{article}
\usepackage[utf8]{inputenc}
\usepackage{polski}
\usepackage{geometry}
\geometry{a4paper, margin=1in}
\begin{document}
\begin{center}
\Large\textbf{Harmonogram RoboDay}
\large\textbf{Zespół Szkół Łączności}
\end{center}
\vspace{1cm}
\begin{center}
\begin{tabular}{|l|l|}
\hline
\textbf{Wydarzenie} & \textbf{Godzina} \\
\hline
Powitanie gości, rozpoczęcie RD & 10.00-10.15 \\
\hline
1. promocja PP i WIiT & 10.15.-10.30 \\
\hline
informacje o konkursach & 10.30-10.40 \\
\hline
2. CPU vs GPU vs TPU vs DPU vs QPU, czyli czym jest to PU (Processing Unit) & 12.10.-12.25 \\
\hline
oglądanie robotów & 11.20-11.40 \\
\hline
np. wyświetlacz 7 segemntowy & 11.45-12.05 \\
\hline
1. kampus z anegdotami & 10.50-11.10 \\
\hline
1 konkurs indywidualny & 12.25-13.00 \\
\hline
2 konkurs indywidualny & 12.25-13.00 \\
\hline
4 konkurs zespołowy & 12.25-13.00 \\
\hline
jak uczyć kreatywności & 13.05-13.20 \\
\hline
zkończenie,rozdania nagród & 13.20-13.35 \\
\hline
sprzątanie (roboty i stoły staramy się zawinąć prędzej?) & 13:30-14:00 \\
\hline
\end{tabular}
\end{center}
\end{document}